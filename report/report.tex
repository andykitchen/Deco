\documentclass[]{article}

\usepackage{parskip}
\usepackage{url}
\usepackage{amssymb}

\usepackage{listings}
\lstloadlanguages{Haskell}
\lstnewenvironment{code}
    {\lstset{}%
      \csname lst@SetFirstLabel\endcsname}
    {\csname lst@SaveFirstLabel\endcsname}
    \lstset{
      basicstyle=\small\ttfamily,
      flexiblecolumns=false,
      basewidth={0.5em,0.45em},
      literate={+}{{$+$}}1 {/}{{$/$}}1 {*}{{$*$}}1 {=}{{$=$}}1
               {>}{{$>$}}1 {<}{{$<$}}1 {\\}{{$\lambda$}}1
               {\\\\}{{\char`\\\char`\\}}1
               {->}{{$\rightarrow$}}2 {>=}{{$\geq$}}2 {<-}{{$\leftarrow$}}2
               {<=}{{$\leq$}}2 {=>}{{$\Rightarrow$}}2 
               {\ .}{{$\circ$}}2 {\ .\ }{{$\circ$}}2
               {>>}{{>>}}2 {>>=}{{>>=}}2
               {|}{{$\mid$}}1
               {forall}{{$\forall$}}2
               {>>=}{{$\gg\!=$}}3 {>>}{{$\gg$}}2
               {()}{{$\square$}}2
    }

\newcommand{\ignore}[1]{}
\newcommand{\function}[1]{\texttt{#1}}
\newcommand{\type}[1]{\texttt{#1}}

\title{Technical Report: Implementing \& Designing the Deco Programming Language}
\author{Andy Kitchen}

\date{2011-10-03}

\begin{document}

\maketitle

\section{Overview}

The programming language Deco is a impure functional programming language.
this report discusses the design process of Deco and it's practical
implementation in Haskell.

The task of building an interpreter for an impure~functional~language inside
a~pure~one is an interesting task. Putting great stress on the abstraction
used to represent state and interaction with the outside world.

While implementing Deco several high-level and distinctive features of
functional programming have been used. Including monad~transformers,
parser~combinators, rank-2~types and delimited~continuations. They have also
been made to work in unison, leading to an interesting synthesis. High-level
features such as these are often described from a very theoretical perspective
and in isolation. This report aims to present the theoretical and the
practical together; and explore the interactions of different features.

\subsection{Running and Compiling}

Deco uses the Haskell Cabal build system to build itself and manage its
dependencies. First install the Haskell platform. Then run the following
commands:

\lstset{language=bash}
\begin{verbatim}
$ cabal install transformers haskeline parsec CC-delcont
$ make
$ ./dist/build/deco/deco example/delimited.deco 
\end{verbatim}

You should see this output if everything was built correctly:

\begin{verbatim}
test case passed: got 117.0
test case passed: got 60.0
test case passed: got 121.0
test case passed: got "ab"
\end{verbatim}

\section{Monads \& Monad Transformers}
\subsection{Monads}
To do input and output in a pure functional language one must put it into
some kind of abstraction so that side effects alter the semantics of the
language in a controlled and manageable way \cite{PeytonJones:2001}.

In Haskell monads are used to handle input and output. Impure values are
wrapped inside a particular monad called the \type{IO} monad. While the theory
of monads is rich and deep, for many purposes the \type{IO} monad can be seen
as a tag indicating that this particular value is influenced by the outside
world. For example the type of \function{readLn} (that reads a line of input)
is \type{IO String} instead of a plain \type{String} indicating that it is a
value obtained from the outside world.

Unfortunately the return value of \function{readLn} is useless without some
way to operate on a value of type \type{IO String} Monads are equipped with a
binding operation in Haskell called `\function{>>=}' which specialised for the
IO monad has type: `\type{$\forall$~a~b.~IO~a~->~(a~->~IO~b)~->~IO~b}'. So to
operate on a value inside the IO monad, one must supply a function that will
be called by an unknown mechanism with the unwrapped value. It must also
promise to return a value that is wrapped up again; indicating that the
function's output is also influenced by the outside world.

While the \type{IO} monad in Haskell is magical, because it cannot
be implemented inside the language itself, there are many
more non-magical monads that can be written using Haskell that are
still very useful.

One such monad is the \type{Maybe} monad. It allows one to cleanly represent
functions that may not return a value. In this case the \type{Maybe} tag can
be interpreted as saying that this value is influenced by computations that
may not have generated a value and so may not actually be present.

\subsection{Transforming Monads}
\label{discussion:monadtrans}

Monads can contain very useful features and allow one to represent
certain patterns of computation very cleanly. But the question is
what if we want several of those features all together?

One desirable thing to do with monad would be to compose its features with
other monads. This can be realised by a monad transformer. A normal monad has
kind `\type{* -> *}' meaning that it is a type that operates on types for
example one can have an \type{IO Int} or \type{IO String}. Monad transformers
have a higher kinded type: `\type{(* -> *) -> * -> *}'. Specifically they take
a \emph{monad} as a type argument and yield a \emph{new monad} that has
been augmented by the monad transformer.

Because the augmented monad generated by a monad transformer is also a monad
this process can be repeated to create a `stack' of monad transformers. Each
transformer on different levels of the stack providing more functionality.

\subsection{ProgramEnv Monad}

In Deco a stack of monad transformers is used to represent the program
environment, simplified but similar to techniques in \cite{Liang:Unknown1}. In
a way the program environment monad indicates that the wrapped value has been
influenced by the stateful execution of Deco code.

\section{Delimited Continuations}
\label{discussion:delcont}

Delimited continuations are a variant of classical continuations. However they
are much more useful in application programming because their effects are more
local \cite{Oleg:2011}. They can also be called composable continuations
because of this.

In a stack based computation model a classical continuation captures the
entire stack and program counter. When the continuation is invoked the current
stack is replaced with the saved stack and the program counter also restored.
When creating a delimited continuation, one point on the stack is marked with
a `prompt'. Then later in the execution the segment of stack between the mark
and the top can be cut-off and saved with execution returned the point the
mark was created. When a delimited continuation is invoked the saved
stack segment is glued on top of the current stack.

Delimited continuations can only be created from prompts marking places higher
in the stack. Code higher in the stack has full control over what prompts are
available later. This means that delimited continuations have more `local'
effects. Delimited continuations naturally control only a small part of
program execution.

\subsection{Implementation in terms of Classical Continuations}
Delimited continuations can be represented using classical continuations. What
follows is a scheme program that implements delimited continuations in terms
of \function{call/cc}. Because scheme has a formal semantics, this is also a
formal proof:

\begin{verbatim}
(define ret-points '())

(define (push-ret ret)
  (set! ret-points (cons ret ret-points)))

(define (pop-ret)
  (let ((ret (car ret-points)))
    (set! ret-points (cdr ret-points))
    ret))

(define (reset* thunk)
  (call/cc (lambda (ret-point)
    (push-ret ret-point)
    (let ((val (thunk))
          (ret (pop-ret)))
      (ret val)))))

(define (shift* handler)
  (call/cc (lambda (shift-cont)
    (let ((val (handler (lambda args
                 (reset* (lambda () (apply shift-cont args)))))))
      ((pop-ret) val)))))

(define-syntax reset
  (syntax-rules ()
    ((_ ?e ?f ...) (reset* (lambda () ?e ?f ...)))))

(define-syntax shift
  (syntax-rules ()
    ((_ ?k ?e ?f ...) (shift* (lambda (?k) ?e ?f ...)))))
\end{verbatim}

\section{Rank-2 Types}
\label{discussion:rank2}

Rank-2 types are types that have a universal quantifier limited in scope to
the left hand side of a function arrow \cite{Wikibook:Haskell}. They are not
natively available in Haskell but are available as an extension in GHC.

Rank-2 types are used by the \type{ST} monad and the \type{CC} monad. This
allows certain values to be created inside the monad but prevented from
escaping. For example the ST monad represents stateful computations that look
pure from the outside, like an in-place quick-sort. One can create mutable
references inside the ST monad, however these references must be prevented
from escaping the context of the monad. This can be achieved by using Rank-2
types.

Rank-2 have interesting treatments in System-F and interesting interpretations
in terms of the Curry-Howard isomorphism.

\subsection{Complex Properties}
\label{discussion:unsafecast}

There are some correct programs for which it is either extremely
complex or impossible to correctly type. In these situations,
there is a backdoor into the type system called \function{unsafeCoerce}
with type \type{a -> b} (for all \type{a} and \type{b}).

In Deco, `prompts' created by the \type{CCT} monad cannot escape. However they
can appear inside the state of a lower \type{StateT} monad, because they can
be stored inside Deco variables. In certain situations these values could
escape. However because the \function{runProgram} function evaluates to
\type{IO ()} and this is the only provided way to unwrap the \type{ProgramEnv}
monad, escape can never happen. Therefore a judiciously used
\function{unsafeCoerce} must be placed into the type of the Deco primitives
that produce prompt values.

\section{Deco Architecture}

\input{../Evaluate.lhs}

\bibliographystyle{plain}
\bibliography{deco}
\end{document}
